\documentclass[11pt, a4paper]{article}

\usepackage[utf8]{inputenc}
\usepackage[T1]{fontenc}
\usepackage[ngerman]{babel}
\usepackage{amsmath}
\usepackage{amssymb}
\usepackage{framed}
\usepackage{graphicx}
\usepackage{hyperref}
\usepackage[
    a4paper,
    left=2.5cm,
    right=2.5cm,
    top=2cm,
    bottom=2.5cm
]{geometry}
\usepackage{enumitem}
\usepackage[most]{tcolorbox}
\usepackage{tikz}
\usetikzlibrary{shapes.geometric, positioning, decorations.pathmorphing, shapes.misc}

\pagestyle{empty}

% Punkte optional machen: bei leerem #2 keine Punkte anzeigen
\newcommand{\aufgabe}[2]{%
  \section*{\Large\bfseries Aufgabe #1%
  \if\relax\detokenize{#2}\relax\else \hfill\normalfont\normalsize(#2 Punkte)\fi}%
  \vspace{-1.5ex}
}

\setlist[enumerate,1]{label=(\alph*)}
\setlist[enumerate,2]{label=(\roman*)}

\tikzset{
    infoBubble/.style={circle, draw, thick, minimum size=1.5cm, align=center},
    dashedBubble/.style={circle, dashed, draw, thick, minimum size=2.5cm},
    saveDateRibbon/.style={
        tape,
        tape bend top=none,
        tape bend bottom=none,
        draw, thick, fill=white,
        text=black, font=\small\bfseries, align=center
    }
}

\begin{document}

\noindent
\begin{minipage}[t]{0.6\textwidth}
    \vspace{0pt}
    \includegraphics[height=1.5cm]{kit_logo.png} \\ % Logo (optional)
    {\Large\bfseries Lineare Algebra 2 für die Fachrichtung Informatik} \\[0.5em]
    {\large\bfseries Inoffizielle Transkription der Aufgabensammlung}
    {(1-1, 3-1, 4-2, 7-1, 8-2, 10-2, 11-1, 11-2, 12-1, 12-3, 13-2)}
\end{minipage}%
\begin{minipage}[t]{0.4\textwidth}
    \vspace{0pt}
    \raggedleft
    \small
    \textbf{Institut für Algebra und Geometrie} \\
    PD. Dr. Stefan Kühnlein \\
    Maximilian Wackenhuth, M. Sc. \\
    \vspace{0.5em}
    \textbf{Lösung formatiert von:} \\
    \href{https://github.com/fregeh}{fregeh} \\
    \vspace{1em}
    \textbf{Sommersemester 2025}
\end{minipage}

\vspace{0.5cm}
\hrule

% =========================
% AUFGABE 1
% =========================
\aufgabe{1}{}
\begin{enumerate}
  \item Es sei $\Phi:\mathbb{R}^n\to\mathbb{R}^n$ ein Endomorphismus, welcher die Gleichung
  \[
    \Phi^3+10\Phi=7\Phi^2
  \]
  erfüllt. Zeigen Sie, dass $\Phi$ diagonalisierbar ist. Nutzen Sie dafür Folgerung 9.1.9, um $\mathbb{R}^n$ in Eigenräume von $\Phi$ zu zerlegen.
  \begin{framed}\end{framed}

  \item Geben Sie ein Beispiel für einen Endomorphismus $\Psi:\mathbb{R}^3\to\mathbb{R}^3$ an, dessen charakteristisches Polynom
  \[
    f(X)=X^3-7X^2
  \]
  ist, der aber nicht diagonalisierbar ist.
  \begin{framed}\end{framed}

  \item Es sei $\Theta:\mathbb{R}^n\to\mathbb{R}^n$ ein diagonalisierbarer Endomorphismus, der nur die Eigenwerte $1,-1$ und $2$ besitzt. Bestimmen Sie
  \[
    \Theta^4-5\Theta^2+5\,\mathrm{id}_{\mathbb{R}^n}.
  \]
  \begin{framed}\end{framed}
\end{enumerate}

% =========================
% AUFGABE 2
% =========================
\aufgabe{2}{}
Bestimmen Sie die Jordansche Normalform der Matrix
\[
A=
\begin{pmatrix}
-3 & -6 & 0 & -1 & 3\\
 1 &  3 & 2 & -3 & -5\\
 0 & -1 & -2 &  4 & 5\\
 0 & -3 & -6 &  6 & 6\\
 0 &  2 & 4 & -4 & -4
\end{pmatrix}\in\mathbb{R}^{5\times 5},
\]
sowie eine Matrix $S$, so dass $S^{-1}AS$ diese Jordansche Normalform ist.
\begin{framed}\end{framed}

% =========================
% AUFGABE 3
% =========================
\aufgabe{3}{}
Sei $A\in\mathbb{C}^{5\times 5}$ mit $\mathrm{Rang}(A)=2$ und $\mathrm{Spur}(A)=0$.
\begin{enumerate}
  \item Wie viele Jordankästchen zum Eigenwert $0$ besitzt die Jordansche Normalform von $A$?
  \begin{framed}\end{framed}

  \item Geben Sie alle Möglichkeiten für die Jordansche Normalform von $A$ an. Unterscheiden Sie dabei die beiden Fälle, dass entweder $0$ der einzige Eigenwert von $A$ ist oder es neben $0$ noch (mindestens) einen Eigenwert $\lambda\neq 0$ gibt.
  \begin{framed}\end{framed}

  \item Begründen Sie, wieso $A$ genau dann diagonalisierbar ist, wenn $\mathrm{Spur}(A^2)\neq 0$ gilt.
  \begin{framed}\end{framed}
\end{enumerate}

% =========================
% AUFGABE 4
% =========================
\aufgabe{4}{}
\begin{enumerate}
  \item Bestimmen Sie für den Untervektorraum $U\subset\mathbb{R}^5$ eine Orthonormalbasis bezüglich des Standardskalarproduktes,
  \[
    U=\left\langle
    \begin{pmatrix}0\\0\\1\\0\\0\end{pmatrix},
    \begin{pmatrix}1\\0\\1\\0\\0\end{pmatrix},
    \begin{pmatrix}2\\1\\1\\0\\2\end{pmatrix},
    \begin{pmatrix}2\\1\\0\\2\\3\end{pmatrix}
    \right\rangle .
  \]
  \begin{framed}\end{framed}

  \item Ergänzen Sie die gefundene Orthonormalbasis von $U$ zu einer Orthonormalbasis von $\mathbb{R}^5$.
  \begin{framed}\end{framed}
\end{enumerate}

% =========================
% AUFGABE 5
% =========================
\aufgabe{5}{}
Wir betrachten im euklidischen Vektorraum $\mathbb{R}^5$ (mit dem Standardskalarprodukt) die Vektoren
\[
v_1:=\begin{pmatrix}0\\0\\1\\0\\0\end{pmatrix},\quad
v_2:=\begin{pmatrix}1\\0\\1\\0\\0\end{pmatrix},\quad
v_3:=\begin{pmatrix}2\\1\\1\\0\\2\end{pmatrix},\quad
v_4:=\begin{pmatrix}2\\1\\0\\2\\3\end{pmatrix},\quad
u:=\begin{pmatrix}0\\1\\0\\1\\0\end{pmatrix},\quad
w:=\begin{pmatrix}0\\-2\\0\\0\\1\end{pmatrix}.
\]
Es sei $V_4:=\langle v_1,v_2,v_3,v_4\rangle$ und $V_3:=\langle v_1,v_2,v_3\rangle$.
\begin{enumerate}
  \item Bestimmen Sie den Abstand $d(u,V_4)$.
  \begin{framed}\end{framed}

  \item Bestimmen Sie den Abstand $d\bigl(u+\langle w\rangle,\,V_3\bigr)$ und die Lotfußpunkte des Lots zwischen $u+\langle w\rangle$ und $V_3$.
  \begin{framed}\end{framed}
\end{enumerate}

% =========================
% AUFGABE 6
% =========================
\aufgabe{6}{}
Es seien $V$ und $W$ euklidische Vektorräume mit $\dim(V)>0$, und $\Phi:V\to W$ eine injektive lineare Abbildung. Zeigen Sie, dass die folgenden Aussagen äquivalent sind:
\begin{enumerate}
  \item Für alle $x,y\in V\setminus\{0\}$ gilt $\angle(x,y)=\angle(\Phi(x),\Phi(y))$.
  \item Für alle $x,y\in V$ gilt $x\perp y\ \Rightarrow\ \Phi(x)\perp \Phi(y)$.
  \item Für alle $x,y\in V$ gilt $\|x\|=\|y\|\ \Rightarrow\ \|\Phi(x)\|=\|\Phi(y)\|$.
  \item Es gibt eine reelle Zahl $r>0$ so, dass für alle $x\in V$ gilt $\|\Phi(x)\|=r\,\|x\|$.
  \item Es gibt eine reelle Zahl $r>0$ und eine lineare Isometrie $\Psi:V\to W$ so, dass $\Phi=r\,\Psi$.
\end{enumerate}
\emph{Hinweis:} Was ist das Skalarprodukt von $x+y$ mit $x-y$?
\begin{framed}\end{framed}

% =========================
% AUFGABE 7
% =========================
\aufgabe{7}{}
Bestimmen Sie für alle $a,b\in\mathbb{R}$ alle linearen Isometrien $\Psi:\mathbb{R}^3\to\mathbb{R}^3$, die
\[
\Psi\!\begin{pmatrix}2\\0\\1\end{pmatrix}=\begin{pmatrix}2\\1\\0\end{pmatrix}
\qquad\text{und}\qquad
\Psi\!\begin{pmatrix}3\\3\\0\end{pmatrix}=\begin{pmatrix}1\\ a\\ b\end{pmatrix}
\]
erfüllen.
\begin{framed}\end{framed}

% =========================
% AUFGABE 8
% =========================
\aufgabe{8}{}
Sei $A\in \mathrm{SO}(4)$.
\begin{enumerate}
  \item Benutzen Sie die Ähnlichkeit von $A$ zu einer Matrix in Isometrie-Normalform, um zu zeigen, dass das charakteristische Polynom von $A$ von folgender Gestalt mit zwei reellen Parametern $c_1,c_2\in[-1,1]$ ist:
  \[
    \mathrm{CP}_A(X)=X^4-2(c_1+c_2)X^3+(4c_1c_2+2)X^2-2(c_1+c_2)X+1.
  \]
  \begin{framed}\end{framed}

  \item Nun habe $A$ das charakteristische Polynom
  \[
    X^4-2X^3+3X^2-2X+1.
  \]
  Bestimmen Sie die Isometrie-Normalform von $A$.
  \begin{framed}\end{framed}
\end{enumerate}

% =========================
% AUFGABE 9
% =========================
\aufgabe{9}{}
Der reelle Vektorraum $\mathbb{R}^3$ sei mit dem Skalarprodukt
\[
\langle x,y\rangle = x^\top\!\cdot
\begin{pmatrix}
1 & 0 & 1\\
0 & 2 & 3\\
1 & 3 & 6
\end{pmatrix} y
\]
versehen. Finden Sie alle $\alpha\in\mathbb{R}$, für die die Abbildung $\Phi:\mathbb{R}^3\to\mathbb{R}^3$, $v\mapsto A\,v$ mit
\[
A=
\begin{pmatrix}
2 & -2 & -3\\
\alpha & -1 & \alpha-4\\
0 & 2 & 5
\end{pmatrix}
\]
bezüglich des Skalarproduktes $\langle\cdot,\cdot\rangle$ selbstadjungiert ist. Stellen Sie für diese $\alpha$ eine Orthonormalbasis aus Eigenvektoren von $\Phi$ auf.
\begin{framed}\end{framed}

% =========================
% AUFGABE 10
% =========================
\aufgabe{10}{}
Im euklidischen Vektorraum $\mathbb{R}^3$ sei $f:\mathbb{R}^3\to\mathbb{R}^3$ eine Drehung, gegeben durch
\[
f\!\begin{pmatrix}2\\0\\1\end{pmatrix}=\begin{pmatrix}2\\1\\0\end{pmatrix},\qquad
f\!\begin{pmatrix}-1\\2\\1\end{pmatrix}=\tfrac{1}{3}\begin{pmatrix}-2\\1\\7\end{pmatrix}.
\]
\begin{enumerate}
  \item Bestimmen Sie die Drehachse von $f$.
  \begin{framed}\end{framed}
  \item Bestimmen Sie eine Orthonormalbasis der Drehebene von $f$.
  \begin{framed}\end{framed}
  \item Bestimmen Sie die euklidische Normalform $\tilde A$ von $f$.
  \begin{framed}\end{framed}
  \item Geben Sie eine Basis von $\mathbb{R}^3$ an, bezüglich der die Abbildungsmatrix von $f$ gleich der Normalform $\tilde A$ ist.
  \begin{framed}\end{framed}
\end{enumerate}

% =========================
% AUFGABE 11 (neu)
% =========================
\aufgabe{11}{}
\begin{enumerate}
  \item Es sei
  \[
    A:=
    \begin{pmatrix}
      0 & 0 & 1 & 3\\
      0 & 0 & 3 & 1\\
     -1 & -3 & 0 & 0\\
     -3 & -1 & 0 & 0
    \end{pmatrix}.
  \]
  Geben Sie zwei orthogonale 2\,-dimensionale Unterräume $U_1,U_2\subset \mathbb{R}^4$ an, so dass $AU_1\subseteq U_1$ und $AU_2\subseteq U_2$.
  \begin{framed}\end{framed}

  \item Es sei nun $A\in\mathbb{R}^{n\times n}$ eine symmetrische Matrix und $v\in\mathbb{R}^n$ ein Eigenvektor. Zeigen Sie, dass dann
  \[
    \left\langle \left\{ \binom{v}{0},\,\binom{0}{v} \right\} \right\rangle \subseteq \mathbb{R}^{2n}
  \]
  unter der Matrix
  \[
    \begin{pmatrix}
      0 & A\\
      -A & 0
    \end{pmatrix}\in\mathbb{R}^{2n\times 2n}
  \]
  invariant ist.
  \begin{framed}\end{framed}
\end{enumerate}

\vspace{0.5em}

\end{document}
