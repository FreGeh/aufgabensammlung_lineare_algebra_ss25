\documentclass[11pt, a4paper]{article}

\usepackage[utf8]{inputenc}
\usepackage[T1]{fontenc}
\usepackage[ngerman]{babel}
\usepackage{amsmath}
\usepackage{amssymb}
\usepackage{framed}
\usepackage{graphicx}
\usepackage{hyperref}
\usepackage[
    a4paper,
    left=2.5cm,
    right=2.5cm,
    top=2cm,
    bottom=2.5cm
]{geometry}
\usepackage{enumitem}
\usepackage[most]{tcolorbox}
\usepackage{tikz}
\usetikzlibrary{shapes.geometric, positioning, decorations.pathmorphing, shapes.misc}

\pagestyle{empty}
\setlength{\parindent}{0pt}  % kein Erstzeileneinzug
\setlength{\parskip}{0.5\baselineskip}  % optional: Absatzabstand
% Punkte optional machen: bei leerem #2 keine Punkte anzeigen
\usepackage{hyperref}
\usepackage{bookmark} % stabilere PDF-Lesezeichen
\hypersetup{
  colorlinks=true,
  linkcolor=blue,
  urlcolor=blue,
  pdfauthor={fregeh},
  pdftitle={Lineare Algebra 2 – Aufgabensammlung mit Lösungen},
  bookmarksopen=true,
  bookmarksopenlevel=2,
  bookmarksnumbered=true
}
\setcounter{secnumdepth}{2}
\setcounter{tocdepth}{2}
% optional: \pdfstringdefDisableCommands{\def\langle{<}\def\rangle{>}}

% ==== Zähler & Makros für Aufgaben/Lösungen mit Gegenlinks ====
\newcounter{aufg}    % zählt Aufgaben
\newcounter{sol}     % zählt Lösungen (in gleicher Reihenfolge)

% \aufgabe{Titeltext wie "1 (ÜB1-A1)"}{Punktzahl(optional leer)}
\newcommand{\aufgabe}[2]{%
  \refstepcounter{aufg}%
  \section{Aufgabe #1%
    \if\relax\detokenize{#2}\relax\else\hfill\normalfont\normalsize(#2 Punkte)\fi}%
  \label{aufgabe:\theaufg}%
  \par\vspace{.5\baselineskip}%
}

% \loesung[optionaler Zusatz wie "ÜB1-A1"]
\newcommand{\loesung}[1][]{%
  \refstepcounter{sol}%
  \subsection{Lösung zu Aufgabe \thesol%
    \if\relax\detokenize{#1}\relax\else\ (\textit{#1})\fi}%
  \label{loes:\thesol}%
  \noindent\hyperref[aufgabe:\thesol]{\small[← zurück zur Aufgabe (\autopageref*{aufgabe:\thesol})]}%
  \par\vspace{.5\baselineskip}%
}

\setlist[enumerate,1]{label=(\alph*)}
\setlist[enumerate,2]{label=(\roman*)}

\tikzset{
    infoBubble/.style={circle, draw, thick, minimum size=1.5cm, align=center},
    dashedBubble/.style={circle, dashed, draw, thick, minimum size=2.5cm},
    saveDateRibbon/.style={
        tape,
        tape bend top=none,
        tape bend bottom=none,
        draw, thick, fill=white,
        text=black, font=\small\bfseries, align=center
    }
}

\begin{document}

\noindent
\begin{minipage}[t]{0.6\textwidth}
    \vspace{0pt}
    \includegraphics[height=1.5cm]{kit_logo.png} \\ % Logo (optional)
    {\Large\bfseries Lineare Algebra 2 für die Fachrichtung Informatik} \\[0.5em]
    {\large\bfseries Inoffizielle Transkription der Aufgabensammlung}
    {(1-1, 3-1, 4-2, 7-1, 8-2, 10-2, 11-1, 11-2, 12-1, 12-3, 13-2)}
\end{minipage}%
\begin{minipage}[t]{0.4\textwidth}
    \vspace{0pt}
    \raggedleft
    \small
    \textbf{Institut für Algebra und Geometrie} \\
    PD. Dr. Stefan Kühnlein \\
    Maximilian Wackenhuth, M. Sc. \\
    \vspace{0.5em}
    \textbf{Lösungen formatiert von:} \\
    \href{https://github.com/fregeh}{fregeh} \\
    \vspace{1em}
    \textbf{Sommersemester 2025}
\end{minipage}

\vspace{0.5cm}
\hrule

\tableofcontents
\bigskip
\hrule

% =========================
% AUFGABE 1
% =========================
\aufgabe{1 (ÜB1-A1)}{}
\begin{enumerate}
  \item Es sei $\Phi:\mathbb{R}^n\to\mathbb{R}^n$ ein Endomorphismus, welcher die Gleichung
  \[
    \Phi^3+10\Phi=7\Phi^2
  \]
  erfüllt. Zeigen Sie, dass $\Phi$ diagonalisierbar ist. Nutzen Sie dafür Folgerung 9.1.9, um $\mathbb{R}^n$ in Eigenräume von $\Phi$ zu zerlegen.

  \item Geben Sie ein Beispiel für einen Endomorphismus $\Psi:\mathbb{R}^3\to\mathbb{R}^3$ an, dessen charakteristisches Polynom
  \[
    f(X)=X^3-7X^2
  \]
  ist, der aber nicht diagonalisierbar ist.

  \item Es sei $\Theta:\mathbb{R}^n\to\mathbb{R}^n$ ein diagonalisierbarer Endomorphismus, der nur die Eigenwerte $1,-1$ und $2$ besitzt. Bestimmen Sie
  \[
    \Theta^4-5\Theta^2+5\,\mathrm{id}_{\mathbb{R}^n}.
  \]

\end{enumerate}

% =========================
% AUFGABE 2
% =========================
\aufgabe{2 (ÜB3-A1)}{}
Bestimmen Sie die Jordansche Normalform der nilpotenten Matrix
\[
A=
\begin{pmatrix}
-3 & -6 & 0 & -1 & 3\\
 1 &  3 & 2 & -3 & -5\\
 0 & -1 & -2 &  4 & 5\\
 0 & -3 & -6 &  6 & 6\\
 0 &  2 & 4 & -4 & -4
\end{pmatrix}\in\mathbb{R}^{5\times 5},
\]
sowie eine Matrix $S$, so dass $S^{-1}AS$ diese Jordansche Normalform ist.

% =========================
% AUFGABE 3
% =========================
\aufgabe{3 (ÜB4-A2)}{}
Sei $A\in\mathbb{C}^{5\times 5}$ mit $\mathrm{Rang}(A)=2$ und $\mathrm{Spur}(A)=0$.
\begin{enumerate}
  \item Wie viele Jordankästchen zum Eigenwert $0$ besitzt die Jordansche Normalform von $A$?

  \item Geben Sie alle Möglichkeiten für die Jordansche Normalform von $A$ an. Unterscheiden Sie dabei die beiden Fälle, dass entweder $0$ der einzige Eigenwert von $A$ ist oder es neben $0$ noch (mindestens) einen Eigenwert $\lambda\neq 0$ gibt.

  \item Begründen Sie, wieso $A$ genau dann diagonalisierbar ist, wenn $\mathrm{Spur}(A^2)\neq 0$ gilt.
\end{enumerate}

% =========================
% AUFGABE 4
% =========================
\aufgabe{4 (ÜB7-A1)}{}
\begin{enumerate}
  \item Bestimmen Sie für den Untervektorraum $U\subset\mathbb{R}^5$ eine Orthonormalbasis bezüglich des Standardskalarproduktes,
  \[
    U=\left\langle
    \begin{pmatrix}0\\0\\1\\0\\0\end{pmatrix},
    \begin{pmatrix}1\\0\\1\\0\\0\end{pmatrix},
    \begin{pmatrix}2\\1\\1\\0\\2\end{pmatrix},
    \begin{pmatrix}2\\1\\0\\2\\3\end{pmatrix}
    \right\rangle .
  \]

  \item Ergänzen Sie die gefundene Orthonormalbasis von $U$ zu einer Orthonormalbasis von $\mathbb{R}^5$.
\end{enumerate}

% =========================
% AUFGABE 5
% =========================
\aufgabe{5 (ÜB8-A2)}{}
Wir betrachten im euklidischen Vektorraum $\mathbb{R}^5$ (mit dem Standardskalarprodukt) die Vektoren ($v_1,\dots,v_4$ sind aus vorheriger Aufgabe)
\[
v_1:=\begin{pmatrix}0\\0\\1\\0\\0\end{pmatrix},\quad
v_2:=\begin{pmatrix}1\\0\\1\\0\\0\end{pmatrix},\quad
v_3:=\begin{pmatrix}2\\1\\1\\0\\2\end{pmatrix},\quad
v_4:=\begin{pmatrix}2\\1\\0\\2\\3\end{pmatrix},\quad
u:=\begin{pmatrix}0\\1\\0\\1\\0\end{pmatrix},\quad
w:=\begin{pmatrix}0\\-2\\0\\0\\1\end{pmatrix}.
\]
Es sei $V_4:=\langle v_1,v_2,v_3,v_4\rangle$ und $V_3:=\langle v_1,v_2,v_3\rangle$.
\begin{enumerate}
  \item Bestimmen Sie den Abstand $d(u,V_4)$.

  \item Bestimmen Sie den Abstand $d\bigl(u+\langle w\rangle,\,V_3\bigr)$ und die Lotfußpunkte des Lots zwischen $u+\langle w\rangle$ und $V_3$.
\end{enumerate}

% =========================
% AUFGABE 6
% =========================
\aufgabe{6 (ÜB10-A2)}{}
Es seien $V$ und $W$ euklidische Vektorräume mit $\dim(V)>0$, und $\Phi:V\to W$ eine injektive lineare Abbildung. Zeigen Sie, dass die folgenden Aussagen äquivalent sind:
\begin{enumerate}
  \item Für alle $x,y\in V\setminus\{0\}$ gilt $\angle(x,y)=\angle(\Phi(x),\Phi(y))$.
  \item Für alle $x,y\in V$ gilt $x\perp y\ \Rightarrow\ \Phi(x)\perp \Phi(y)$.
  \item Für alle $x,y\in V$ gilt $\|x\|=\|y\|\ \Rightarrow\ \|\Phi(x)\|=\|\Phi(y)\|$.
  \item Es gibt eine reelle Zahl $r>0$ so, dass für alle $x\in V$ gilt $\|\Phi(x)\|=r\,\|x\|$.
  \item Es gibt eine reelle Zahl $r>0$ und eine lineare Isometrie $\Psi:V\to W$ so, dass $\Phi=r\,\Psi$.
\end{enumerate}
\emph{Hinweis:} Was ist das Skalarprodukt von $x+y$ mit $x-y$?

% =========================
% AUFGABE 7
% =========================
\aufgabe{7 (ÜB11-A1)}{}
Bestimmen Sie für alle $a,b\in\mathbb{R}$ alle linearen Isometrien $\Psi:\mathbb{R}^3\to\mathbb{R}^3$, die
\[
\Psi\!\begin{pmatrix}2\\0\\1\end{pmatrix}=\begin{pmatrix}2\\1\\0\end{pmatrix}
\qquad\text{und}\qquad
\Psi\!\begin{pmatrix}3\\3\\0\end{pmatrix}=\begin{pmatrix}1\\ a\\ b\end{pmatrix}
\]
erfüllen.

% =========================
% AUFGABE 8
% =========================
\aufgabe{8 (ÜB11-A2)}{}
Sei $A\in \mathrm{SO}(4)$.
\begin{enumerate}
  \item Benutzen Sie die Ähnlichkeit von $A$ zu einer Matrix in Isometrie-Normalform, um zu zeigen, dass das charakteristische Polynom von $A$ von folgender Gestalt mit zwei reellen Parametern $c_1,c_2\in[-1,1]$ ist:
  \[
    \mathrm{CP}_A(X)=X^4-2(c_1+c_2)X^3+(4c_1c_2+2)X^2-2(c_1+c_2)X+1.
  \]

  \item Nun habe $A$ das charakteristische Polynom
  \[
    X^4-2X^3+3X^2-2X+1.
  \]
  Bestimmen Sie die Isometrie-Normalform von $A$.
\end{enumerate}

% =========================
% AUFGABE 9
% =========================
\aufgabe{9 (ÜB12-A1)}{}
Der reelle Vektorraum $\mathbb{R}^3$ sei mit dem Skalarprodukt
\[
\langle x,y\rangle = x^\top\!\cdot
\begin{pmatrix}
1 & 0 & 1\\
0 & 2 & 3\\
1 & 3 & 6
\end{pmatrix} y
\]
versehen. Finden Sie alle $\alpha\in\mathbb{R}$, für die die Abbildung $\Phi:\mathbb{R}^3\to\mathbb{R}^3$, $v\mapsto A\,v$ mit
\[
A=
\begin{pmatrix}
2 & -2 & -3\\
\alpha & -1 & \alpha-4\\
0 & 2 & 5
\end{pmatrix}
\]
bezüglich des Skalarproduktes $\langle\cdot,\cdot\rangle$ selbstadjungiert ist. Stellen Sie für diese $\alpha$ eine Orthonormalbasis aus Eigenvektoren von $\Phi$ auf.

% =========================
% AUFGABE 10
% =========================
\aufgabe{10 (ÜB12-A3)}{}
Im euklidischen Vektorraum $\mathbb{R}^3$ sei $f:\mathbb{R}^3\to\mathbb{R}^3$ eine Drehung, gegeben durch
\[
f\!\begin{pmatrix}2\\0\\1\end{pmatrix}=\begin{pmatrix}2\\1\\0\end{pmatrix},\qquad
f\!\begin{pmatrix}-1\\2\\1\end{pmatrix}=\tfrac{1}{3}\begin{pmatrix}-2\\1\\7\end{pmatrix}.
\]
\begin{enumerate}
  \item Bestimmen Sie die Drehachse von $f$.

  \item Bestimmen Sie eine Orthonormalbasis der Drehebene von $f$.

  \item Bestimmen Sie die euklidische Normalform $\tilde A$ von $f$.

  \item Geben Sie eine Basis von $\mathbb{R}^3$ an, bezüglich der die Abbildungsmatrix von $f$ gleich der Normalform $\tilde A$ ist.

\end{enumerate}

% =========================
% AUFGABE 11 (neu)
% =========================
\aufgabe{11 (ÜB13-A2)}{}
\begin{enumerate}
  \item Es sei
  \[
    A:=
    \begin{pmatrix}
      0 & 0 & 1 & 3\\
      0 & 0 & 3 & 1\\
     -1 & -3 & 0 & 0\\
     -3 & -1 & 0 & 0
    \end{pmatrix}.
  \]
  Geben Sie zwei orthogonale 2\,-dimensionale Unterräume $U_1,U_2\subset \mathbb{R}^4$ an, so dass $AU_1\subseteq U_1$ und $AU_2\subseteq U_2$.

  \item Es sei nun $A\in\mathbb{R}^{n\times n}$ eine symmetrische Matrix und $v\in\mathbb{R}^n$ ein Eigenvektor. Zeigen Sie, dass dann
  \[
    \left\langle \left\{ \binom{v}{0},\,\binom{0}{v} \right\} \right\rangle \subseteq \mathbb{R}^{2n}
  \]
  unter der Matrix
  \[
    \begin{pmatrix}
      0 & A\\
      -A & 0
    \end{pmatrix}\in\mathbb{R}^{2n\times 2n}
  \]
  invariant ist.

\end{enumerate}


\vspace{0.5em}

% =========================
% LÖSUNGEN (separat am Ende)
% =========================
\section{\Large\bfseries Lösungen}

\loesung[ÜB1-A1]
\begin{enumerate}
  \item
  Wir wissen, dass für $f(X)=X^3-7X^2+10X=X(X-5)(X-2)$ gilt $f(\Phi)=0$. Nach 9.1.9 folgt
  \[
    \mathbb{R}^n=\ker(\Phi)\oplus\ker\!\big((\Phi-5\mathrm{id})(\Phi-2\mathrm{id})\big).
  \]
  Für $x\in\ker\!\big((\Phi-5\mathrm{id})(\Phi-2\mathrm{id})\big)$ gilt
  \[
    (\Phi-5\mathrm{id})(\Phi-2\mathrm{id})\Phi(x)
     =\Phi(\Phi-5\mathrm{id})(\Phi-2\mathrm{id})(x)=\Phi(0)=0,
  \]
  also ist dieser Kern $\Phi$-invariant. Setze
  \[
    \varphi:=\Phi\big|_{\ker((\Phi-5\mathrm{id})(\Phi-2\mathrm{id}))}.
  \]
  Dann ist $g(\varphi)=0$ für $g(X)=(X-2)(X-5)$, und wieder nach 9.1.9
  \[
    \ker\!\big((\Phi-5\mathrm{id})(\Phi-2\mathrm{id})\big)
      =\ker(\varphi-5\mathrm{id})\oplus\ker(\varphi-2\mathrm{id}).
  \]
  Insgesamt also
  \[
    \mathbb{R}^n=\ker(\Phi)\oplus\ker(\varphi-5\mathrm{id})\oplus\ker(\varphi-2\mathrm{id}).
  \]
  Für $x\in\ker(\varphi-5\mathrm{id})$ ist $\Phi(x)=\varphi(x)=5x$, und für
  $x\in\ker(\varphi-2\mathrm{id})$ ist $\Phi(x)=2x$. Wählt man Basen $B_0$,
  $B_5$, $B_2$ der drei Kerne, so ist $\Phi$ bezüglich
  $B=B_0\cup B_5\cup B_2$ diagonal. Damit ist $\Phi$ diagonalisierbar.

  \medskip\noindent\textit{Intuition: Minimalpolynom zerfällt; der Raum spaltet in Eigenräume, auf denen $\Phi$ nur skaliert.}

  \item
  Betrachte
  \[
    A=\begin{pmatrix}
      7&0&0\\
      0&0&1\\
      0&0&0
    \end{pmatrix}.
  \]
  Dann ist $\operatorname{Spec}(A)=\{0,7\}$, $\dim(E(A,0))=1$ und
  $\dim(E(A,7))=1$. Also ist
  $E(A,0)\oplus E(A,7)\neq\mathbb{R}^3$ und $A$ somit nicht diagonalisierbar,
  obwohl $\chi_A(X)=X^3-7X^2$.

  \medskip\noindent\textit{Intuition: Ein Jordanblock zum Nullwert verkleinert den Eigenraum; daher reicht die Eigenraumdimension nicht.}

  \item
  Für $v=v_{1}+v_{-1}+v_{2}$ mit $v_\lambda$ im Eigenraum zu $\lambda\in\{1,-1,2\}$ gilt
  \[
  (\Theta^4-5\Theta^2+5\mathrm{id})(v)
  =\sum_{\lambda\in\{1,-1,2\}}(\lambda^4-5\lambda^2+5)v_\lambda
  =v.
  \]
  Also $\Theta^4-5\Theta^2+5\,\mathrm{id}=\mathrm{id}$.

  \medskip\noindent\textit{Intuition: Polynom wirkt skalarm auf Eigenräumen; Koeffizienten heben sich passend auf.}
\end{enumerate}

\loesung[ÜB3-A1]
\textbf{Schritt 1: Nilpotenz und Kerndimensionen.}
Direkte Rechnung ergibt
\[
A^2=
\begin{pmatrix}
3 & 9 & 6 & 3 & 3\\
0 & 0 & 0 & 0 & 0\\
-1 & -3 & -2 & -1 & -1\\
-3 & -9 & -6 & -3 & -3\\
2 & 6 & 4 & 2 & 2
\end{pmatrix},
\qquad
A^3=0 .
\]
Alle Zeilen von \(A^2\) sind Vielfache von \((1,3,2,1,1)\), also \(\operatorname{rang}(A^2)=1\) und
\[
\ker(A^2)=\{x\in\mathbb R^5:\ x_1+3x_2+2x_3+x_4+x_5=0\}
=\left\langle
\begin{pmatrix}-3\\1\\0\\0\\0\end{pmatrix},
\begin{pmatrix}-2\\0\\1\\0\\0\end{pmatrix},
\begin{pmatrix}-1\\0\\0\\1\\0\end{pmatrix},
\begin{pmatrix}-1\\0\\0\\0\\1\end{pmatrix}
\right\rangle.
\]
Ferner
\[
\ker(A)=\left\langle
\begin{pmatrix}4\\-2\\1\\0\\0\end{pmatrix},
\begin{pmatrix}7\\-2\\0\\-3\\2\end{pmatrix}
\right\rangle,
\]
also \(\dim\ker A=2\), \(\dim\ker A^2=4\), \(\dim\ker A^3=5\).
Damit (über \(d_k=\dim\ker A^k-\dim\ker A^{k-1}\)) gilt:
\(d_1=2,\ d_2=2,\ d_3=1\).
Folglich JNF \(=J_3(0)\oplus J_2(0)\).

\medskip
\textbf{Schritt 2: Jordan\-ketten.}

\emph{3er-Kette:} Wir wählen \(w_3=e_1\) und setzen
\[
w_2:=A w_3=\begin{pmatrix}-3\\1\\0\\0\\0\end{pmatrix},\qquad
w_1:=A^2 w_3=\begin{pmatrix}3\\0\\-1\\-3\\2\end{pmatrix},
\]
sodass \(A w_1=0,\ A w_2=w_1,\ A w_3=w_2\).

\medskip
\emph{2er-Kette:} Wir wählen \(v_2\in\ker(A^2)\setminus\ker(A)\)
\[
v_2:=\begin{pmatrix}-1\\0\\0\\0\\1\end{pmatrix},\qquad
v_1:=A v_2=\begin{pmatrix}6\\-6\\5\\6\\-4\end{pmatrix}.
\]
Dann \(A v_1=0\) (also \(v_1\in\ker A\)) und \(A v_2=v_1\) (also $v_2\notin\ker A$).

\medskip
\textbf{Schritt 3: Ähnlichkeitstransformation.}
Mit
\[
S=\bigl(w_1\ \big|\ w_2\ \big|\ w_3\ \big|\ v_1\ \big|\ v_2\bigr)=
\begin{pmatrix}
  3 & -3 & 1 &  6 & -1\\
  0 &  1 & 0 & -6 &  0\\
 -1 &  0 & 0 &  5 &  0\\
 -3 &  0 & 0 &  6 &  0\\
  2 &  0 & 0 & -4 &  1
\end{pmatrix},\qquad
J=\operatorname{diag}\!\big(J_3(0),J_2(0)\big)
\]
gilt \(AS=SJ\) (damit \(S^{-1}AS=J\)), denn spaltenweise:
\[
A w_1=0,\ A w_2=w_1,\ A w_3=w_2,\qquad A v_1=0,\ A v_2=v_1,
\]
was exakt den Superdiagonalen-1en in \(J\) entspricht.

\medskip\noindent
\emph{Intuition:} Kerndimensionen liefern die Blockgrößen, die Ketten aus \(e_1\) und einem zweiten Startvektor erzeugen die Basis für \(J_3(0)\) bzw.\ \(J_2(0)\).
\loesung[ÜB4-A2]
\begin{enumerate}
  \item
  Es gilt $\dim E(A,0)=5-\mathrm{Rang}(A)=3$. Daher besitzt die Jordansche
  Normalform von $A$ genau \emph{drei} Jordankästchen zum Eigenwert $0$.

  \medskip\noindent\textit{Intuition: Rang-Dimensionssatz liefert Kerndimension; diese zählt die Null-Kästchen.}

  \item
  \textbf{Fall 1:} $\operatorname{Spec}(A)=\{0\}$.
  Möglich (bis auf Blockreihenfolge):
  \[
    J_2(0)\oplus J_2(0)\oplus J_1(0)
    \quad\text{oder}\quad
    J_3(0)\oplus J_1(0)\oplus J_1(0).
  \]
  \textbf{Fall 2:} Es existiert $\lambda\neq 0$. Wegen $\mathrm{Spur}(A)=0$
  dann auch $-\lambda$ Eigenwert. Also
  \[
    \operatorname{diag}(\lambda,\,-\lambda,\,0,\,0,\,0).
  \]

  \medskip\noindent\textit{Intuition: Spur $0$ erzwingt Paarbildung $\lambda,-\lambda$; restliche Dimensionen werden durch Nullen aufgefüllt.}

  \item
\emph{Hin:} (b).  Ist $\mathrm{Spur}(A^2)\neq 0$, so hat $A$ einen von $0$ verschiedenen Eigenwert $\lambda$.
Wegen $\mathrm{Spur}(A)=0$ ist dann auch $-\lambda$ Eigenwert. 
Da $\operatorname{rang}(A)=2$, kann es im $0$-Teil keine nichttrivialen Jordanblöcke geben 
(denn ein $J_2(0)$ würde bereits den Rang um $1$ erhöhen). 
Ebenso kann es für $\pm\lambda$ keine Jordanblöcke der Größe $>1$ geben, 
da dies den Rang auf mindestens $3$ treiben würde. 
Folglich ist $A$ diagonalisierbar mit Spektrum $\{\lambda,-\lambda,0,0,0\}$.

Umgekehrt: Ist $A$ diagonalisierbar mit Eigenwerten $\{\lambda,-\lambda,0,0,0\}$, so
$\mathrm{Spur}(A^2)=\lambda^2+(-\lambda)^2=2\lambda^2\neq 0$.

\medskip\noindent\textit{Intuition: Nichtverschwindende Quadratsumme signalisiert echte von $0$ verschiedene Eigenwerte und damit Diagonalisierbarkeit.}
\end{enumerate}

\loesung[ÜB7-A1]
\begin{enumerate}
  \item
  Gram–Schmidt auf $c_1,c_2,c_3,c_4$ mit
  \[
  c_1=\begin{pmatrix}0\\0\\1\\0\\0\end{pmatrix},\;
  c_2=\begin{pmatrix}1\\0\\1\\0\\0\end{pmatrix},\;
  c_3=\begin{pmatrix}2\\1\\1\\0\\2\end{pmatrix},\;
  c_4=\begin{pmatrix}2\\1\\0\\2\\3\end{pmatrix}.
  \]
  Es ergibt sich
  \[
  b_1=c_1,\qquad
  b_2=c_2-\langle b_1,c_2\rangle b_1=\begin{pmatrix}1\\0\\0\\0\\0\end{pmatrix},
  \]
  \[
  b_3=c_3-\langle c_3,b_1\rangle b_1-\langle c_3,b_2\rangle b_2
      =\begin{pmatrix}0\\1\\0\\0\\2\end{pmatrix},\ \ \|b_3\|=\sqrt5,
  \]
  \[
  b_4=c_4-\langle c_4,b_1\rangle b_1-\langle c_4,b_2\rangle b_2
      -\frac{\langle c_4,b_3\rangle}{\|b_3\|^2}b_3
      =\begin{pmatrix}0\\-\tfrac{2}{5}\\0\\2\\\tfrac{1}{5}\end{pmatrix},
      \ \ \|b_4\|^2=\tfrac{21}{5}.
  \]
  Eine Orthonormalbasis von $U$ ist damit
  \[
  \Bigl\{\,b_1,\ b_2,\ \tfrac{1}{\sqrt5}b_3,\ \tfrac{\sqrt5}{\sqrt{21}}\,b_4\,\Bigr\}.
  \]

  \medskip\noindent\textit{Intuition: Projektionen entfernen sukzessive Komponenten, danach normieren.}

  \item
  Ergänze mit $c_5=\begin{pmatrix}0\\0\\0\\1\\0\end{pmatrix}$. Orthogonalisieren:
  \[
  b_5=c_5-\langle c_5,b_1\rangle b_1-\langle c_5,b_2\rangle b_2
      -\frac{\langle c_5,b_3\rangle}{\|b_3\|^2}b_3
      -\frac{\langle c_5,b_4\rangle}{\|b_4\|^2}b_4
      =\begin{pmatrix}0\\\tfrac{4}{21}\\0\\\tfrac{1}{21}\\-\tfrac{2}{21}\end{pmatrix}.
  \]
  Da $\|b_5\|^2=\tfrac{1}{21}$, ist
  \[
   \widehat b_5=\frac{1}{\sqrt{21}}\begin{pmatrix}0\\4\\0\\1\\-2\end{pmatrix}.
  \]
  Eine ONB von $\mathbb{R}^5$ ist somit
  \[
  \left\{\, b_{1},\; b_{2},\; \tfrac{1}{\sqrt5}\,b_{3},\;
  \tfrac{\sqrt5}{\sqrt{21}}\,b_{4},\;
  \tfrac{1}{\sqrt{21}}\begin{pmatrix}0\\4\\0\\1\\-2\end{pmatrix}\right\}.
  \]

  \medskip\noindent\textit{Intuition: Letzten Richtungsvektor orthogonalisieren und normieren — fertig.}
\end{enumerate}

\loesung[ÜB8-A2]
\begin{enumerate}
  \item
  Es gilt $d(u,V_4)=\|\pi_{V_4^\perp}(u)\|$. Aus Aufgabe 4: 
  $V_4^\perp=\left\langle \frac{1}{\sqrt{21}}\begin{pmatrix}0\\4\\0\\1\\-2\end{pmatrix}\right\rangle$.
  Daher
  \[
  \pi_{V_4^\perp}(u)=\frac{1}{21}\,\left\langle \begin{pmatrix}0\\1\\0\\1\\0\end{pmatrix},
  \begin{pmatrix}0\\4\\0\\1\\-2\end{pmatrix}\right\rangle
  \begin{pmatrix}0\\4\\0\\1\\-2\end{pmatrix}
  =\frac{5}{21}\begin{pmatrix}0\\4\\0\\1\\-2\end{pmatrix},
  \]
  und somit
  \[
  d(u,V_4)=\bigl\|\pi_{V_4^\perp}(u)\bigr\|=\frac{5}{\sqrt{21}}.
  \]

  \medskip\noindent\textit{Intuition: Abstand $=$ Norm der Projektion auf $V_4^\perp$; dieses ist eindimensional.}

  \item
  $d(u+\langle w\rangle,V_3)=\|\pi_{(V_3+\langle w\rangle)^\perp}(u)\|$.
  Eine ONB von $V_3$ ist
  \[
  B=\left\{\begin{pmatrix}0\\0\\1\\0\\0\end{pmatrix},
  \begin{pmatrix}1\\0\\0\\0\\0\end{pmatrix},
  \frac{1}{\sqrt5}\begin{pmatrix}0\\1\\0\\0\\2\end{pmatrix}\right\},
  \]
  und damit eine ONB von $V_3+\langle w\rangle$ nach Ergänzung mit
  $\frac{1}{\sqrt5}\begin{pmatrix}0\\-2\\0\\0\\1\end{pmatrix}$.
  Projektion:
  \[
  \pi_{V_3+\langle w\rangle}(u)
  =\frac{1}{5}\begin{pmatrix}0\\1\\0\\0\\2\end{pmatrix}
  -\frac{2}{5}\begin{pmatrix}0\\-2\\0\\0\\1\end{pmatrix}
  =\begin{pmatrix}0\\1\\0\\0\\0\end{pmatrix}.
  \]
  Also
  \[
  \pi_{(V_3+\langle w\rangle)^\perp}(u)
  =u-\pi_{V_3+\langle w\rangle}(u)
  =\begin{pmatrix}0\\0\\0\\1\\0\end{pmatrix},
  \]
  und $d(u+\langle w\rangle,V_3)=1$.

  Schreibe $\pi_{V_3+\langle w\rangle}(u)=w'+v$ mit $w'\in\langle w\rangle$, $v\in V_3$:
  \[
  w'=-\frac{2}{5}\begin{pmatrix}0\\-2\\0\\0\\1\end{pmatrix},\qquad
  v=\frac{1}{5}\begin{pmatrix}0\\1\\0\\0\\2\end{pmatrix}.
  \]
  Lotfußpunkte:
  \[
  u-w'=\begin{pmatrix}0\\\tfrac{1}{5}\\0\\1\\\tfrac{2}{5}\end{pmatrix},
  \qquad 
  v=\frac{1}{5}\begin{pmatrix}0\\1\\0\\0\\2\end{pmatrix}.
  \]

  \medskip\noindent\textit{Intuition: Zerlege die Projektion in $V_3$- und $\langle w\rangle$-Anteil; der Rest gibt den Abstand.}
\end{enumerate}

\loesung[ÜB10-A2]
\underline{(i)$\Rightarrow$(ii)} Klar. \\
\underline{(ii)$\Rightarrow$(iii)} $\langle x+y,x-y\rangle=\|x\|^2-\|y\|^2$.
Bei $\|x\|=\|y\|$ folgt $0=\langle\Phi(x+y),\Phi(x-y)\rangle
=\|\Phi(x)\|^2-\|\Phi(y)\|^2$. \\
\underline{(iii)$\Rightarrow$(iv)} Sei $x\neq 0$, setze $r:=\|\Phi(x)\|/\|x\|$.
Zu $y\neq 0$ wähle $\lambda>0$ mit $\|\lambda y\|=\|x\|$. Dann
\[
\|\Phi(x)\|=\|\Phi(\lambda y)\|=\lambda\|\Phi(y)\|\;\Rightarrow\;
\|\Phi(y)\|=r\|y\|.
\]
\underline{(iv)$\Rightarrow$(v)} $\Psi(x):=\tfrac1r\,\Phi(x)$ ist Isometrie. (Da $\Phi$ injektiv ist, gilt $\Phi(x)\neq 0$ für $x\neq 0$. 
Damit ist $r=\frac{\|\Phi(x)\|}{\|x\|}$ wohldefiniert und strikt positiv.) \\
\underline{(v)$\Rightarrow$(i)} Winkel bleiben unter Isometrie und einheitlicher Skalierung erhalten.

\medskip\noindent\textit{Intuition: Winkel, Orthogonalität und Normen folgen alle aus dem Skalarprodukt; jede Aussage erzwingt eine feste Skalierung und einen isometrischen Anteil.}

\loesung[ÜB11-A1]
\textbf{Methode 1 (Skalarprodukte und Orthogonalkomplemente).}
Normerhalt liefert $1+a^2+b^2=18$, also $a=4$ und $b=\pm1$. Weiter
$\langle (2,0,1),(3,3,0)\rangle=2+a\Rightarrow a=4$. 
Das Orthogonalkomplement von
$\langle(2,0,1),(3,3,0)\rangle$ ist
$\left\langle\begin{pmatrix}-1\\1\\2\end{pmatrix}\right\rangle$.
\medskip

\textbf{Fall $b=-1$:}
\[
\left\langle\begin{pmatrix}2\\1\\0\end{pmatrix},
\begin{pmatrix}1\\4\\-1\end{pmatrix}\right\rangle^{\!\perp}
=\left\langle\begin{pmatrix}-1\\2\\7\end{pmatrix}\right\rangle.
\]
\textbf{Fall $b=1$:}
\[
\left\langle\begin{pmatrix}2\\1\\0\end{pmatrix},
\begin{pmatrix}1\\4\\1\end{pmatrix}\right\rangle^{\!\perp}
=\left\langle\begin{pmatrix}1\\-2\\7\end{pmatrix}\right\rangle.
\]
Da $\bigl\|(-1,1,2)\bigr\|^2=6$ und $\|( \pm1,\mp2,7)\|^2=54=9\cdot6$,
\[
\Psi\!\left(3\begin{pmatrix}-1\\1\\2\end{pmatrix}\right)
=\pm\begin{pmatrix}b\\-2b\\7\end{pmatrix}.
\]
Insgesamt (für $b\in\{\pm1\}$) entstehen vier Isometrien:
\[
\Psi\!\begin{pmatrix}2\\0\\1\end{pmatrix}=\begin{pmatrix}2\\1\\0\end{pmatrix},\quad
\Psi\!\begin{pmatrix}3\\3\\0\end{pmatrix}=\begin{pmatrix}1\\4\\b\end{pmatrix},\quad
\Psi\!\begin{pmatrix}-1\\1\\2\end{pmatrix}= \pm \frac{1}{3}\begin{pmatrix}b\\-2b\\7\end{pmatrix}.
\]

\smallskip
\textbf{Methode 2 (Drehung/Spiegelung).}
Für $\det(\Psi)=1$ ist $\Psi$ eine Drehung. 
\begin{itemize}
\item \emph{Unterfall $b=+1$:} Drehebene aus
$b_1=(0,-1,1)$ und $b_2=(2,-1,-1)$; Achse $\langle(1,1,1)\rangle$,
Drehwinkel mit
\(
\cos\alpha_1=\dfrac{\langle (1,-1,0),(1,0,-1)\rangle}{\|(1,-1,0)\|^2}=\dfrac12.
\)
\item \emph{Unterfall $b=-1$:} Drehebene aus
$b_1=(0,-1,1)$ und $b_2'=(2,-1,1)$; Achse $\langle(0,1,1)\rangle$,
\(
\cos\alpha_2=\dfrac{7}{9}.
\)
\end{itemize}
Für $\det(\Psi)=-1$ spiegle in der Ebene
$\left\langle\begin{pmatrix}2\\1\\0\end{pmatrix},\begin{pmatrix}1\\a\\b\end{pmatrix}\right\rangle$
und reduziere auf den Drehungsfall.

\medskip\noindent\textit{Intuition: Längen/Winkel fixieren zwei Bilder; das Orthogonalkomplement bestimmt die dritte Richtung. Fallunterscheidung über $b=\pm1$.}

\loesung[ÜB11-A2]
\begin{enumerate}
  \item
  $A$ ist ähnlich zu $\operatorname{diag}(D_\alpha,D_\beta)$ mit
  $D_\theta=\begin{pmatrix}\cos\theta&-\sin\theta\\ \sin\theta&\cos\theta\end{pmatrix}$.
  Also
  \[
  \mathrm{CP}_A(X)=(X^2-2\cos\alpha\,X+1)(X^2-2\cos\beta\,X+1),
  \]
  d.\,h.\ die behauptete Form mit $c_1=\cos\alpha$, $c_2=\cos\beta$.

  \medskip\noindent\textit{Intuition: Produkt zweier Rotationsfaktoren.}

  \item
  Vergleich: $c_1+c_2=1$, $4c_1c_2+2=3\Rightarrow c_1c_2=\tfrac14$.
  Daraus $c_1=c_2=\tfrac12$, also $\alpha=\beta=\tfrac\pi3$ und
  \[
  \operatorname{diag}\!\big(D_{\pi/3},D_{\pi/3}\big)
  =
  \begin{pmatrix}
  \tfrac12&-\tfrac{\sqrt3}{2}&0&0\\
  \tfrac{\sqrt3}{2}&\tfrac12&0&0\\
  0&0&\tfrac12&-\tfrac{\sqrt3}{2}\\
  0&0&\tfrac{\sqrt3}{2}&\tfrac12
  \end{pmatrix}.
  \]

  \medskip\noindent\textit{Intuition: Summen/Produkte der Kosinuswerte fixieren die Winkel.}
\end{enumerate}

\loesung[ÜB12-A1]
\textbf{Selbstadjungiertheit.}
Für alle $x,y$ gelte $\langle Ax,y\rangle=\langle x,Ay\rangle$.
Dies ist äquivalent dazu, dass $FA$ \emph{symmetrisch} ist.

\medskip
\textbf{Rechnung von $FA$.}
\[
FA
=
\begin{pmatrix}1&0&1\\0&2&3\\1&3&6\end{pmatrix}
\begin{pmatrix}2&-2&-3\\ \alpha&-1&\alpha-4\\ 0&2&5\end{pmatrix}
=
\begin{pmatrix}
2&0&2\\
2\alpha&4&2\alpha+7\\
2+3\alpha&7&3\alpha+15
\end{pmatrix}.
\]
Symmetrie verlangt insbesondere $(FA)_{12}=(FA)_{21}$, also $0=2\alpha$.
\[
\Rightarrow\ \boxed{\alpha=0}.
\]

\medskip
\textbf{Spektrum und Eigenräume für $\alpha=0$.}
\[
A=
\begin{pmatrix}2&-2&-3\\ 0&-1&-4\\ 0&2&5\end{pmatrix},\qquad
\chi_A(X)=(X-2)(X-1)(X-3).
\]
\emph{Zu $\lambda=2$:}
\[
\ker(A-2I)=
\ker\begin{pmatrix}0&-2&-3\\ 0&-3&-4\\ 0&2&3\end{pmatrix}
=\left\langle\begin{pmatrix}1\\0\\0\end{pmatrix}\right\rangle.
\]
\emph{Zu $\lambda=1$:}
\[
\ker(A-I)=
\ker\begin{pmatrix}1&-2&-3\\ 0&-2&-4\\ 0&2&4\end{pmatrix}
=\left\langle\begin{pmatrix}-1\\-2\\1\end{pmatrix}\right\rangle.
\]
\emph{Zu $\lambda=3$:}
\[
\ker(A-3I)=
\ker\begin{pmatrix}-1&-2&-3\\ 0&-4&-4\\ 0&2&2\end{pmatrix}
=\left\langle\begin{pmatrix}-1\\-1\\1\end{pmatrix}\right\rangle.
\]

\medskip
\textbf{Normierung bzgl.\ $\langle\cdot,\cdot\rangle$.}
\[
\langle e_1,e_1\rangle=1.
\]
Für $v_1=\begin{pmatrix}-1\\-2\\1\end{pmatrix}$:
\[
Fv_1=\begin{pmatrix}0\\-1\\-1\end{pmatrix},\quad
\langle v_1,v_1\rangle=v_1\cdot(Fv_1)=1.
\]
Für $v_2=\begin{pmatrix}-1\\-1\\1\end{pmatrix}$:
\[
Fv_2=\begin{pmatrix}0\\1\\2\end{pmatrix},\quad
\langle v_2,v_2\rangle=v_2\cdot(Fv_2)=1.
\]
Eigenvektoren zu verschiedenen Eigenwerten sind orthogonal. Eine ONB ist daher
\[
\mathcal{B}=\left\{\,\begin{pmatrix}1\\0\\0\end{pmatrix},\
\begin{pmatrix}-1\\-2\\1\end{pmatrix},\
\begin{pmatrix}-1\\-1\\1\end{pmatrix}\right\}.
\]

\medskip\noindent\textit{Intuition: Selbstadjungiertheit $\Leftrightarrow FA$ symmetrisch; eine Off-Diagonalbedingung fixiert $\alpha$.}

\loesung[ÜB12-A3]
\begin{enumerate}
  \item
  Aus $f(v)-v\in U$ (Drehebene) folgt
  \[
  U=\left\langle\begin{pmatrix}0\\1\\-1\end{pmatrix},
  \begin{pmatrix}1\\-5\\4\end{pmatrix}\right\rangle,\qquad
  U^\perp=\left\langle\begin{pmatrix}1\\1\\1\end{pmatrix}\right\rangle.
  \]
  Die Drehachse ist $\langle(1,1,1)\rangle$.

  \medskip\noindent\textit{Intuition: Differenzen $f(v)-v$ liegen in der Drehebene; deren Orthogonalkomplement ist die Achse.}

  \item
  Wir orthogonalisieren die Basis der Ebene, die wir in (a) bestimmt haben.\\
  Wähle $u_1=\dfrac{1}{\sqrt2}\begin{pmatrix}1\\-1\\0\end{pmatrix}$.\\
  Orthogonalisiere \\
  \[
  \tilde u_2=\begin{pmatrix}1\\-5\\4\end{pmatrix}
  -\frac{\langle(1,-5,4),(1,-1,0)\rangle}{\|u_1\|^2}\,u_1
  =\frac{1}{6}\begin{pmatrix}6\\-1\\-1\end{pmatrix},
  \]
  Normiere \\
  \[
  n_1=\frac{1}{\sqrt{2}}\begin{pmatrix}1\\-1\\0\end{pmatrix} ; 
  n_2=\frac{1}{\sqrt{38}}\begin{pmatrix}6\\-1\\-1\end{pmatrix} 
  \]
  Daraus folgt die Orthonormalbasis
  \[
  \left\{\,
  \frac{1}{\sqrt{2}}\begin{pmatrix}1\\-1\\0\end{pmatrix},
  \frac{1}{\sqrt{38}}\begin{pmatrix}6\\-1\\-1\end{pmatrix}\right\}.
  \]
  \item
  Die Isometrienormalform einer Dreidimensionalen Drehung besteht aus einem Drehkästchen (mit dem Winkel der Drehung) und einer $1$. \\
  Den Winkel der Drehung bestimmen wir, indem wir den Winkel zwischen einem Vektor $v$ auf der Drehebene und seiner Drehung $f(v)$ bestimmen.\\
  $\begin{pmatrix}1\\-1\\0\end{pmatrix} = \begin{pmatrix}2\\0\\1\end{pmatrix} - \begin{pmatrix}1\\1\\1\end{pmatrix}$ 
  ist orthogonal zur Drehachse $\begin{pmatrix}1\\ 1\\ 1\end{pmatrix}$ und somit auf der Drehebene. \\
  $f\begin{pmatrix}1 \\ -1 \\ 0\end{pmatrix}=f\begin{pmatrix}2 \\ 0 \\ 1\end{pmatrix}-f\begin{pmatrix}1 \\ 1 \\ 1\end{pmatrix}=\begin{pmatrix}2 \\ 1 \\ 0\end{pmatrix}-\begin{pmatrix}1 \\ 1 \\ 1\end{pmatrix}=\begin{pmatrix}1 \\ 0 \\ -1\end{pmatrix}$

  Wir suchen also ein $\alpha \in [0,\pi)$ mit:\\
  $\cos(\alpha)=\frac{\langle(1,-1,0),(1,0,-1)\rangle}{\|(1,-1,0)\|\cdot \|(1,0,-1)\| } = \frac 12\implies \alpha = \frac{\pi}{3}$

  Somit folgt:
  \[
  \tilde A=\begin{pmatrix}
  1&0&0\\
  0&\cos(\frac{\pi}{3})&-\sin(\frac{\pi}{3})\\
  0&\cos(\frac{\pi}{3})&\sin(\frac{\pi}{3})
  \end{pmatrix}=
  \begin{pmatrix}
  1&0&0\\
  0&\tfrac12&-\tfrac{\sqrt3}{2}\\
  0&\tfrac{\sqrt3}{2}&\tfrac12
  \end{pmatrix}.
  \]
  \medskip\noindent\textit{Intuition: Identität auf der Achse, Rotation in der Ebene.}

  \item
  \[
  B=\left\{
  \frac{1}{\sqrt3}\begin{pmatrix}1\\1\\1\end{pmatrix},\
  \frac{1}{\sqrt2}\begin{pmatrix}1\\-1\\0\end{pmatrix},\
  \frac{1}{\sqrt6}\begin{pmatrix}1\\1\\-2\end{pmatrix}
  \right\}.
  \]

  \medskip\noindent\textit{Intuition: Normierte Achse plus ONB der Drehebene.}
\end{enumerate}

\loesung[ÜB13-A2]
\begin{enumerate}
  \item
  Man rechnet nach:
  \[
    A\!\begin{pmatrix}1\\1\\0\\0\end{pmatrix}=-4\!\begin{pmatrix}0\\0\\1\\1\end{pmatrix},\quad
    A\!\begin{pmatrix}0\\0\\1\\1\end{pmatrix}=4\!\begin{pmatrix}1\\1\\0\\0\end{pmatrix},
  \]
  \[
    A\!\begin{pmatrix}1\\-1\\0\\0\end{pmatrix}=2\!\begin{pmatrix}0\\0\\1\\-1\end{pmatrix},\quad
    A\!\begin{pmatrix}0\\0\\1\\-1\end{pmatrix}=-2\!\begin{pmatrix}1\\-1\\0\\0\end{pmatrix}.
  \]
  Damit
  \[
    U_1=\left\langle \begin{pmatrix}1\\1\\0\\0\end{pmatrix},
                       \begin{pmatrix}0\\0\\1\\1\end{pmatrix}\right\rangle,\qquad
    U_2=\left\langle \begin{pmatrix}1\\-1\\0\\0\end{pmatrix},
                       \begin{pmatrix}0\\0\\1\\-1\end{pmatrix}\right\rangle
  \]
  sind orthogonal und $A$-invariant.

  \medskip\noindent\textit{Intuition: Symmetrische/antisymmetrische Richtungen koppeln paarweise.}

  \item
  Für $Av=\lambda v$:
  \[
    \begin{pmatrix}0&A\\-A&0\end{pmatrix}\!\binom{v}{0}
      =\binom{0}{-\lambda v},\qquad
    \begin{pmatrix}0&A\\-A&0\end{pmatrix}\!\binom{0}{v}
      =\binom{\lambda v}{0}.
  \]
  Der von $\binom{v}{0}$, $\binom{0}{v}$ erzeugte Unterraum ist damit invariant.

  \medskip\noindent\textit{Intuition: Blockoperator wirkt als Rotation im zweidimensionalen Spannraum des Eigenvektors.}
\end{enumerate}

\end{document}
